\section{Introduction}
If someone tried to sell a Computational Fluid Dynamics software, based on
the Lattice Boltzmann Method to, e.g.\@, a naval industry, he/she 
would probably have to face the argument ``but we successfully use a
software that solves the Navier-Stokes equations for many years now!''.
Indeed, the Navier-Stokes(-Fourier) equations are the main tool of the classical approach
to the CFD. Yet, they bring with them some important \textit{assumptions}, that restrict
their usage in a limited range of applications. The fundamental assumption they
pose is the \textit{continuity of the matter}. In the case of the naval industry,
that has to solve problems mainly in the scales we can see and touch, the matter
appears to be continuous, so the application of the NSE is usually legit. 

Modern engineers, though, need to design devices far smaller than ships, in which
fluids or gases need e.g.\@ to flow through pipes smaller than hair. Such devices
are for example micropumps, micromotors and other Micro-Electro-Mechanical 
Systems~\cite{Karniadakis_Microflows}. Microflows and nanoflows are also faced 
in Chemical Engineering and relevant fields, in devices like microreactors~\cite{Neumann2012},
particle separators~\cite{Karniadakis_Microflows} or in materials with nanopores, 
like in the shale-gas extraction process. 
The continuity assumption breaks down in
such scales, as e.g.\@ the molecules of the used gases are so rare that interact
mainly with the walls and less with other particles. The same holds also for
very low-pressure scenarios as in aero-astronautics applications~\cite{Toschi2005}.

In such problems, LBM can be applied, though the Bhatnagar-Gross-Krook approach, with Single-Relaxation-Time
and bounce-back boundary conditions,
poses some problems because of the rare collisions of the molecules, that is
assumed to be the main mechanism that leads to an equilibrium (thermalisation). Various
modifications have been proposed to solve these problems and some of them are 
presented in this work.